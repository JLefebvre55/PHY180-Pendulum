%% Copyright 2019 Matheus H. J. Saldanha <mhjsaldanha@gmail.com>
%
% This work may be distributed and/or modified under the
% conditions of the LaTeX Project Public License, either version 1.3
% of this license or (at your option) any later version.
% The latest version of this license is in
%   http://www.latex-project.org/lppl.txt
% and version 1.3 or later is part of all distributions of LaTeX
% version 2005/12/01 or later.
%
% This work has the LPPL maintenance status `maintained'.

\documentclass[12pt,a4paper]{article}

% Pacotes para o português.
\usepackage[utf8]{inputenc}
\usepackage[T1]{fontenc}

\usepackage{graphicx}     % Comando \includegraphics
\usepackage{xcolor}       % Comando de cores \textcolor
\usepackage{indentfirst}  % Indenta o primeiro parágrafo de cada seção
\usepackage{url}          % Comandos \url e \href
\usepackage[top=2cm, bottom=2cm, left=2cm, right=2cm]{geometry} % Define as margens do documento
\usepackage{multirow}     % Permite criar tabelas com uma célula ocupando várias linhas
\usepackage{amssymb}      % Símbolos matemáticos
\usepackage{amsmath}      % Ambientes para escrever fórmulas, \begin{align} por exemplo.
\usepackage{caption}      % Para definir o estilo das legendas de figuras e tabelas.
\usepackage{setspace}     % Para definir espaçamento entre linhas. (\onehalfspacing, \singlespacing, \doublespacing)
\usepackage{breakcites}   % Para permitir quebra de linha no meio de citações.
\usepackage{times}        % Fonte Times New Roman
\usepackage{lipsum}       % Para gerar texto temporário. Exemplo: \lipsum \lipsum[1] \lipsum[4-5].
\usepackage{hyperref}     % Faz os links ficarem azuis e clicáveis. Facilita a navegação pelo PDF.

% Comando para marcar o texto para revisão.
% \newcommand{\rev}[1]{\textcolor{red}{#1}}

% Permite escrever aspas normais "text" em vez de ``text''
\usepackage[autostyle]{csquotes}
\MakeOuterQuote{"}

\begin{document}

\doublespacing

\begin{titlepage}
    \begin{center}
        {\large \sc University of Toronto} \\
        {\large \sc Faculty of Applied Siences and Engineering}\\[0.7cm]
        {\small \sc Division of Engineering Science}
        
        \vspace{4cm}

        % Título.
        {\large \sc PHY180F}\\
        \rule{\linewidth}{2pt}
        
        \vspace{0.7em} % Ajuste ao seu gosto
        {\Large \bfseries Pendulum Lab - Final Report}
        \vspace{0.2em} % Ajuste ao seu gosto
        
        \rule{\linewidth}{2pt} \\
        {\small \sc December 9th, 2020}
    \end{center}
    
    \vspace{2.8cm}

    % Assinaturas
    \begin{minipage}{0.43\textwidth}
        \emph{Author:}\\[1cm]
        \rule{0.9\linewidth}{0.3mm}\\
        Jayden Lefebvre
    \end{minipage}
    \hspace{1cm}
    \begin{minipage}{0.43\textwidth}
        \emph{Professor:}
        \hspace{1.5cm} Dr. Joseph Thywissen\\
        \emph{Teaching Assistant: }
        Mr. Saeed Oghbaey
    \end{minipage}

    \vfill
\end{titlepage}


\pagestyle{empty}
\tableofcontents
\newpage
\setcounter{page}{1}
\pagestyle{plain} % Agora passa a numerar as páginas

\section{Introduction}
\label{section:intro}
\input{text/intro.tex}

\section{Background}
\label{section:background}
\input{text/background.tex}

\section{Objectives}
\label{section:objectives}
\input{text/objectives.tex}

\section{Method}
\label{section:method}
\input{text/method.tex}


\newpage

\bibliographystyle{apalike}
\bibliography{bibliography.bib}

\end{document}